%!TEX root = 0_thesis.tex

\section{The Model}
\label{sec:ds:model}

This section presents the concepts within the model, which provides an effective overview of the design of SC tools. We present each of our 4 sub-models separately, listing both the decisions and related requirements.

For each concept in the DS sub-model, we will discuss its origins, at what point it was added to the DS model, and if applicable where it was added from. Sub-concepts will indicate their parent in the DS sub-model where applicable. 

Some decisions in the model are used to provide added structure to a sub-model, as well as being a decision that needs to be made. If a decision has only decisions as it's children, the sub-decision may be considered as \emph{both} a decision and an option.

\subsection{Functional Sub-model}

Figure \ref{ds:fig:functional} shows a graphical representation of the relationships between the design decisions and options in the functional DS sub-model.

\begin{figure}[htb]
	\centering
	\includegraphics[width=\textwidth,height=0.9\textheight,keepaspectratio]{images/ds/functional_submodel}
	\caption{The functional DS sub-model}
	\label{ds:fig:functional}
\end{figure}

\clearpage

The functional sub-model is grouped by an adaptation activities of the SC lifecycle that are identified by Mehandjiev et al. \cite{Mehandjiev2012}. They identify a set of stages that make up the SC process, similar to the dynamic SC lifecycle presented by Silva et al. \cite{Silva2008}. Silva's lifecycle is presented in Section \ref{bg:sec:lifecycle}. In the rest of this discussion, we will refer to Silva et al.'s lifecycle as \emph{L1} and Mehandjiev et at.'s lifecycle as \emph{L2}.


\emph{L1} is broader in scope than \emph{L2}, since it also includes the creation and publication of the service into the SC tool, which is necessary before composition can occur. However, \emph{L2} is broken down into much more fine-grained activities within each main stage of the lifecycle. A mapping between the two is shown in Figure \ref{ds:fig:lifecyle_mapping}.

\begin{figure}[htb]
	\centering
	\includegraphics[width=0.6\textwidth]{images/ds/lifecycle_comparison}
	\caption{The mapping between the stages presented in the the two different views of SC lifecycles: \emph{L1} \cite{Silva2008} and \emph{L2} \cite{Mehandjiev2012}.}
	\label{ds:fig:lifecyle_mapping}
\end{figure}

There are two levels of granularity across the two lifecycles: a high level in \emph{L1} and the stages in \emph{L2}, and a lower level of granularity in the activities of \emph{L2}. We feel that neither of these two levels are particularly useful in the DS model, since the lower level is too specific and the higher level is not specific enough. Thus, we decided to use a set of decisions that was in the middle ground between these two levels of granularity:
\begin{itemize*}
	\item Creation \& Publication
	\item Specification
	\item Request
	\item Discovery
	\item Composition
	\item Verification \& Validation
	\item Annotation \& Deployment
	\item Execution
\end{itemize*}

\subsubsection{Creation \& Publication}

\textbf{Creation \& Publication} \\ \emph{Decision} \\  Lit review \cite{Silva2008}

Creation and publication is a pre-requisite for the composition process in that in order for there to be component services to be available to be composed they must first be created and then published \cite{Silva2008}.

This area is a prime example of where the tool review is deficient -- the creation and publication process was not visible in any of the tools that we reviewed, hence no options are currently present in the model to solve the creation and publication decision.

\subsubsection{Specification} 

\textbf{Specification} \\ \emph{Decision} \\ Prior model collation \cite{Mehandjiev2012}

The specification phase is the stage at which the user of the SC tool determines what they need from the composite that they seek to create. This stage is present in \emph{L2}, but isn't discussed in \emph{L1}, presumably because it is a stage at which the tool can have little impact on the user who is carrying out the specification.

\textbf{No specification} \\ \emph{Option} -- solves \textbf{Specification} \\ Tool review

None of the prior models considered any options for assisting with the specification process, nor was it discussed in the literature, nor supported in any of the tool reviews. This leads us to believe that this process is one that is part of the SC process, but one that is not explicitly supported by the tool, instead it is left to the user.

\subsubsection{Request}

\textbf{Request} \\ \emph{Decision} \\ Prior model collation \cite{Mehandjiev2012}

The request stage of the process is presented in \emph{L1}, and focuses on the mechanism that the user uses to translate their requirements (i.e. the output of the specification phase) into being able to perform discovery, which is the next phase in the composition process. The request phase is meant to exist in an SC tool that provides fully automated SC, which has been cited as being unrealistic with current service-oriented technologies \cite{Vulcu2008}.

\textbf{Search} \\ \emph{Decision} -- solves \textbf{Request} \\ Prior model collation \cite{Grammel2010}

Searching covers both searching for components that could be used in a composition and searching for composites that have already been created to meet the user's needs. This decision was identified in the requirements gathering process, where participants indicated that this would be their preferred mechanism for finding what they need in a SC tool.

Searching wasn't identified as a concept from the tool review, which was due to the limited number of components that are available in the tool -- if there are only a limited number of components, a search isn't required.

\textbf{Search metrics} \\ \emph{Decision} -- solves \textbf{Search} \\ Requirements gathering study

This decision represents the different ways that the user of the tool can perform the search. It was added to group together options related to search that were elicited in the requirements gathering process.

\textbf{Search by name} \\ \emph{Option} -- solves \textbf{Search metrics} \\ Requirements gathering study

Searching by name was identified in the requirements gathering process as a property of the services available in the tool that the user could use to search for services that they want to use. The name of the service is a prominent attribute that we would expect to be one that users would want to use for search.

\textbf{Search by function} \\ \emph{Option} -- solves \textbf{Search metrics} \\ Requirements gathering study

Searching by function was identified in the requirements gathering process as a property of services available in the tool that the user could use to search for services that they want to use. The function of the service is a more unusual aspect of the services that could be used to search, as it is difficult to determine how this function would be represented.

\subsubsection{Discovery}

\textbf{Discovery} \\ \emph{Decision} \\ Prior model collation \cite{Mehandjiev2012}

The discovery phase of the SC lifecycle is the point at which the tool presents the user with the components that they want to use in composition, or the results of the search that they performed in the last phase. In this sub-model, discovery is broken down into two further decisions: the discovery mechanism and the entities that are discoverable using the tool.

\paragraph{Discovery Mechanism}

\textbf{Discovery mechanism} \\ \emph{Decision} \\ Tool review

Discovery mechanisms are the ways that the user can discover services using the tool -- either components to use in composition or composites that have already been created that they might be able to use or modify.

This decision was created to group together the suggestions that were found in prior DS models, and the discovery mechanisms identified in the tool review.

\textbf{Browsing} \\ \emph{Decision} -- solves \textbf{Discovery mechanism} \\ Tool review

Browsing was identified in the tool review as a common mechanism used by SC tools to present services to the user. This decision was created as a separate parent decision to group together the different types of browsing that can occur, as well as the options that browsing can have.

\textbf{Browsing \& filtering} \\ \emph{Option} -- solves \textbf{Browsing} \\ Tool review

Browsing and filtering represents the option to filter the list of services that are being viewed as part of browsing, regardless of the representation being used during the browsing process. This option was also added as part of the tool review, with a few of the SC tools offering this feature.

\textbf{Browse a list} \\ \emph{Option} -- solves \textbf{Browsing} \\ Tool review

Browsing a list is the standard mechanism for browsing services that was found in the tool review, although the representations of the lists differed somewhat. The discussion of the attributes that services presented is discussed further in the service sub-model.

\textbf{Grouping} \\ \emph{Decision} -- solves \textbf{Browsing} \\ Requirements gathering study

Grouping of the services was identified by the participants in our requirements gathering study as an important mechanism for browsing through the available services in the tool.

\textbf{Grouping metrics} \\ \emph{Decision} -- solves \textbf{Grouping} \\ Requirements gathering study

Various different metrics were identified by the participants in our requirements gathering study, and this decision was added to group these together under one decision.

\textbf{Group by function} \\ \emph{Option} -- solves \textbf{Grouping metrics} \\ Prior model collation \cite{Grammel2010}

A popular approach to grouping services is grouping by the function that the service performs, which could also be described as splitting them into categories. This was identified in the DS model by Grammel et al. \cite{Grammel2010}, as well as being suggested by participants in the requirements gathering study.

\textbf{Group by location} \\ \emph{Option} -- solves \textbf{Grouping metrics} \\ Requirements gathering study

Grouping the services by their location was a concept suggested by participants in the requirements gathering study. This relies on the architecture of the tool providing services from more than one location. In the case of the prototype used in the requirements study, the users of the prototype were able to browse through and acquire services, so this ``location'' was based on whether they had already acquired that service or not.

It could also represent services that are based on the Web (e.g. search, lookups, social media, etc.), vs. those that are available on the device (e.g. accelerometer, SMS, etc.)

\textbf{Group by cost} \\ \emph{Option} -- solves \textbf{Grouping metrics} \\ Requirements gathering study

Grouping by cost relies on there being some monetary value attached to the service. This was also identified in the requirements gathering study, although its identification is unexpected because the study did not present them with any SC tools that presented a monetary value with the services it provided.

\textbf{Group by service provider} \\ \emph{Option} -- solves \textbf{Grouping metrics} \\ Tool review

Grouping by service provider relies on a set of services all being provided by the same provider. For example, \emph{Atooma} provides groups of services to be used with Twitter, Facebook, and Google Drive.

\textbf{Group by rating} \\ \emph{Option} -- solves \textbf{Grouping metrics} \\ Requirements gathering study

Grouping by rating relies on users of the tool being able to assign a star rating to services that can be used with the tool. Whilst rating was available in some of the tools in the tool review, grouping by rating was not. This option was identified by participants in our requirements gathering study, and we suggest there are two possible reasons for it being identified by our participants:
\begin{itemize*}
	\item They combined other elements from the DS model -- ratings and grouping.
	\item They took influence from other design spaces (for example apps and app stores).
\end{itemize*}

\textbf{Group by recently used} \\ \emph{Option} -- solves \textbf{Grouping metrics} \\ Requirements gathering study

Grouping services by how recently they were used wasn't present in any of the tools presented to the participants in the requirements gathering study. This option also presents overlap between different design spaces.

\textbf{Customisation of grouping} \\ \emph{Option} -- solves \textbf{Grouping} \\ Requirements gathering study

Participants in our requirements gathering study suggested that if the tool allowed services to be grouped, then the grouping should be customisable.

\textbf{Manual grouping} \\ \emph{Option} -- solves \textbf{Grouping} \\ Requirements gathering study

Services should be grouped manually by the user, allowing them to put them in any groups that they want.

\textbf{Tagging} \\ \emph{Option} -- solves \textbf{Browsing} \\ Prior model collation \cite{Grammel2010}

Instead of gathering services into groups, tags can be applied to them that categorise them by various properties. Traditionally services would only be able to be in a single group, but they would be able to have multiple tags.

\textbf{Suggestions} \\ \emph{Decision} -- solves \textbf{Discovery mechanism} \\ Prior model collation \cite{Grammel2010}

The system provides the user with some form of suggestion to aid the discovery process.

\textbf{Context-specific suggestions} \\ \emph{Option} -- solves \textbf{Suggestions} \\ Prior model collation \cite{Grammel2010}

Context-specific suggestions are suggestions that are made to the user based upon some particular context change that occurs.

\textbf{Recommendations from friends} \\ \emph{Option} -- solves \textbf{Suggestions} \\ Requirements gathering study

Recommendations from friends relies on some notion of social interaction between users of the tool. This could either piggyback some social network or building a social feature into the tool. This option was identified in the requirements gathering study.

\textbf{``Matching'' components} \\ \emph{Option} -- solves \textbf{Suggestions} \\ Requirements gathering study

Suggesting ``matching components'' relies on the passing of data between components in the composition, and the idea of matching data types in the inputs and outputs of these components. Thus, if the user selects a component with a particular output, the tool can suggest other components whose inputs match with the output of the original selection. This option was identified in the tool review, where participants identified potential frustration when they were allowed to select components whose inputs did not match, but then they could not connect them together.

\textbf{Other recommendations} \\ \emph{Option} -- solves \textbf{Suggestions} \\ Tool review

Other recommendations are recommendations that don't fit within any of the other categories, which were present in various tools that were covered in the tool review.

\paragraph{Discoverable Entities}

\textbf{Discoverable entities} \\ \emph{Decision} \\ Tool review

Discoverable entities are the entities (i.e. services) that the user can discover using the tool.

\textbf{Components} \\ \emph{Option} -- solves \textbf{Discoverable entities} \\ Tool review

Discoving components is mandatory in any SC tool that is not fully automatic, since to be able to compose components together to create a new composite, the user must first be able to discover the components with which they are performing composition.

\textbf{Composites} \\ \emph{Option} -- solves \textbf{Discoverable entities} \\ Tool review

Composite discovery means that the tool allows the user to discover composites -- either created by other users of the tool, or provided as samples by the creator of the tool.

\subsubsection{Composition}

\textbf{Composition} \\ \emph{Decision} \\ Prior model collation \cite{Mehandjiev2012}

The composition stage is the point at which the user of the tool has discovered the components that they wish to use and they are then performing the composition. In practice, this stage overlaps somewhat with the discovery stage, in that the user may discover and compose components one at a time.

There are three different ``types'' of service composition based on the type of object being composed, or how the composition is arranged:
\begin{itemize*}
	\item Process composition
	\item Data composition
	\item Interface composition
\end{itemize*}

\textbf{Process composition} \\ \emph{Option} -- solves \textbf{Composition} \\ Prior model collation \cite{Minhas2012,Daniel2007}

Process composition requires the components in the composition to form a series of processes, which can then be combined together to create one composite process. The components may optionally pass data between one another, which can be manipulated by the other processes. Process composition was identified in the prior model collation stage -- the model created by Minhas et al. \cite{Minhas2012}, which was originally focussed on Mashups. The literature review stage confirmed that process composition is also present in SC as a whole \cite{Daniel2007}.

\textbf{Data composition} \\ \emph{Option} -- solves \textbf{Composition} \\ Prior model collation \cite{Minhas2012,Daniel2007}

Data composition is often thought of as being different to process composition, although they are both structured in the same way. Data composition is based on manipulating data by combining processes to manipulate this data. In mashups, they often rely on data feeds, or APIs \cite{Minhas2012}. As with process composition, data composition was first identified in mashups in Minhas et al.'s model \cite{Minhas2012}, before also being identified in SC in the literature review stage \cite{Daniel2007}.

\textbf{UI composition} \\ \emph{Option} -- solves \textbf{Composition} \\ Prior model collation \cite{Minhas2012,Daniel2007}

User interface (UI) composition allows users to create new interfaces by combining previously created interfaces components, as well as various other interface-related aspects of composition. As with the other two types of composition, this was identified in mashups in the DS model by Minhas et al. \cite{Minhas2012}, before being backed up in the literature review \cite{Daniel2007}

\textbf{UI creation method} \\ \emph{Decision} -- solves \textbf{Composition} \\ Prior model collation

This decision is meant to group together the various different methods for performing UI composition that have been identified in prior DS models. Each of these different methods of UI composition/generation were identified in the prior DS model created by Grammel et al. \cite{Grammel2010}

\textbf{Automatic UI generation} \\ \emph{Option} -- solves \textbf{UI composition method} \\ Prior model collation \cite{Grammel2010}

In automatic UI generation, the interface is generated automatically from some other information. For instance, in Web Services, it is possible to generate a form-like interface from a WSDL description of the service, since the WSDL provides information such as the names of properties and the type of data that they are expecting \cite{Grammel2010}.

\textbf{Webpage composition} \\ \emph{Option} -- solves \textbf{UI composition method} \\ Prior model collation \cite{Minhas2012}

Webpage composition is a specific form for UI composition that allows the user of the tool to browse, compose and edit Web pages or their content in order to create a new, composite Web-based interface \cite{Grammel2010}. Webpage composition normally takes place at the presentation layer (see Section \ref{bg:sec:layers}).

\textbf{Visual UI composition} \\ \emph{Option} -- solves \textbf{UI composition method} \\ Prior model collation \cite{Grammel2010}

Visual UI composition is based on the mechanism that it used to create the UI -- the components are combined visually, by dragging components, etc.

\textbf{Textual UI composition} \\ \emph{Option} -- solves \textbf{UI composition method} \\ Prior model collation \cite{Grammel2010}

Textual UI composition relies on some text-based language to specify the elements and layout of the UI.

\subsubsection{Verification and Validation}

\textbf{Verification \& validation} \\ \emph{Decision} \\ Prior model collation \cite{Mehandjiev2012}

Verification \& validation is the stage in the SC lifecycle at which the user has already composed their service, and they are now seeking to validate that the composite that they have created to ensure that it performs the task that they sought it to complete when they envisioned its creation.

\textbf{Testing} \\ \emph{Decision} -- solves \textbf{Verification \& validation} \\ Prior model collation \cite{Grammel2010}

One mechanism for the user to ensure that their created composite does what it should is to test that composite by executing it before it gets deployed. Testing was identified as a Software Engineering technique in the prior DS model created by Grammel et al. \cite{Grammel2010}.

\textbf{Simulate testing of triggers} \\ \emph{Option} -- solves \textbf{Testing} \\ Requirements gathering study

Of the SC tools in the tool review that supported triggers (components that are activated upon some event-based trigger), none of them support testing. In the requirements gathering study, participants identified that testing of the composites that contain these triggers would be useful and hence is something that should be included in a tool.

\textbf{Test mode for components} \\ \emph{Option} -- solves \textbf{Testing} \\ Requirements gathering study

Another instance of a problem with testing that was identified by participants in our requirements gathering study was components whose execution would have some impact on the ``outside world''. For these components, they indicated that there would need to be some test mode where they could run the composite containing the component without the component affecting the outside world.

\textbf{Dummy test data} \\ \emph{Option} -- solves \textbf{Testing} \\ Requirements gathering study

Similar to having a test mode due to components affecting the outside world, there are components that perform a lookup, for example on the Web. Participants in our requirements gathering study identified that it might be useful to get the component to return some sample data when performing testing, especially if the later components in the composite rely on the initial component returning a particular answer.

\textbf{Debugging} \\ \emph{Option} -- solves \textbf{Testing} \\ Prior model collation \cite{Grammel2010}

Debugging is suggested as a software engineeering technique in Grammel et al.'s prior DS model \cite{Grammel2010}. It is a much more technically oriented option that would normally be in traditional software engineering approaches.

\subsubsection{Annotation and Deployment}

\textbf{Annotation \& deployment} \\ \emph{Decision} \\ Prior model collation \cite{Mehandjiev2012}

Annotation and deployment is the part of the process where the user is able to save the composite that have created so that it can be executed (or edited, shared, etc.) later.

\textbf{Code generation} \\ \emph{Option} -- solves \textbf{Annotation \& deployment} \\ Prior model collation \cite{Pietschmann2010}

Code generation is a deployment method that generates some code (e.g. XML) to represent the composite that can be interpreted and executed by another system. This method is often used in web-based composition tools \cite{Pietschmann2010}.

\textbf{Interpretation} \\ \emph{Option} -- solves \textbf{Annotation \& deployment} \\ Prior model collation \cite{Mehandjiev2012}

Interpretation deploys the composite service within the application in which it is created, and that tool then interprets some internal representation of the composite and hence executes it.

\textbf{Compilation} \\ \emph{Option} -- solves \textbf{Annotation \& deployment} \\ Prior model collation \cite{Pietschmann2010}

Compilation deploys composites by creating a standalone executable application for the composite. This would normally be executable on the platform on which the tool runs.

\subsubsection{Execution}

\textbf{Execution} \\ \emph{Decision} \\ Prior model collation \cite{Mehandjiev2012}

Execution is the final main phase of the composition process, although it can optionally follow repetitions of earlier phases, for example editing and verifying the composite at a later date.

\textbf{In-tool execution} \\ \emph{Option} -- solves \textbf{Execution} \\ Tool review

This is analogous to interpretation, where the execution of the composite takes place inside the tool in which it was created.

\textbf{Create new app} \\ \emph{Option} -- solves \textbf{Execution} \\ Tool review

This is analagous to compilation, where the tool creates an executable application out of a composite, which the user is then able to execute independently of the SC tool.

\subsection{Non-Functional Sub-model}

Figures \ref{ds:fig:nonfunctional1} and \ref{ds:fig:nonfunctional2} show a graphical representation of the non-functional sub-model of the whole DS model. Most of the structure of the non-functional sub-model is based on the prior model created by Aghaee et al. \cite{Aghaee2012}, and provided the general structure for this sub-model of the DS model.

\begin{figure}[htb]
	\centering
	\includegraphics[width=\textwidth,height=0.9\textheight,keepaspectratio]{images/ds/nonfunctional_submodel_1}
	\caption{The first part of the non-functional DS sub-model}
	\label{ds:fig:nonfunctional1}
\end{figure}


\begin{figure}[htb]
	\centering
	\includegraphics[width=\textwidth,height=0.9\textheight,keepaspectratio]{images/ds/nonfunctional_submodel_2}
	\caption{The second part of the non-functional DS sub-model}
	\label{ds:fig:nonfunctional2}
\end{figure}

\clearpage

\subsubsection{Interaction Technique}

\textbf{Interaction technique} \\ \emph{Decision} \\ Prior model collation \cite{Aghaee2012}

Interaction techniques are the mechanisms that the tool provides to allow the user to interact with the composition process, or components and links within the composition. Note that in the majority of cases, the interaction mecahnism also reflects the representation of the composition process, although it can be independent of the representation of the components or links between components.

\textbf{Non-language-based interaction} \\ \emph{Decision} -- solves \textbf{Interaction technique} \\ Prior model collation

This decision focuses in those interaction techniques that do not involve some form of (either visual or textual) language. It was created in the initial collation stage to group together the options providing non-language based interactions.

\textbf{Spreadsheet interaction} \\ \emph{Option} -- solves \textbf{Non-language-based interaction} \\ Prior model collation \cite{Aghaee2012,Minhas2012,Fischer2009}

Interacting with a spreadsheet is an interaction mechanism that has been used in creating mashups \cite{Skrobo2011}, and hence could be extended to work with SC tools in general. Having a spreadsheet interaction mechanism means that the representation of the composition process must also be a spreadsheet.

\textbf{Dialog/form-based interaction} \\ \emph{Option} -- solves \textbf{Non-language-based interaction} \\ Prior model collation \cite{Aghaee2012}

In a dialog/form-based interaction technique, users enter data into a form to control the composition process. Typically the form controls are based on the intputs to the component, and the user is able to connect together the forms represent the different components in the composition \cite{Aghaee2012}.

\textbf{WYSIWYG interaction} \\ \emph{Option} -- solves \textbf{Non-language-based interaction} \\ Prior model collation \cite{Aghaee2012}

WYSIWYG -- ``What you see is what you get'' is based on the premise of presenting the user with the view of the component that they will interact with when the composite service is being executed \cite{Aghaee2012}.

\textbf{Programming by Demonstration (PbD)} \\ \emph{Option} -- solves \textbf{Non-language-based interaction} \\ Prior model collation \cite{Grammel2010,Aghaee2012,Minhas2012,Fischer2009}

Programming by demonstration (PbD) relies on the user of the tool demonstrating the task they want to be performed. The tool then records the steps in this demonstration, and is able to automate them. The macro recorder in the MS Office suite is a popular example of PbD in action.

\textbf{Drag \& drop interaction} \\ \emph{Option} -- solves \textbf{Non-language-based interaction} \\ Prior model collation \cite{Pietschmann2010}

Drag \& drop interaction means that the user interacts with the components in the composition by dragging them around some canvas, and then dropping them in the right position.

\textbf{Programming by Example Modification (PbEM)} \\ \emph{Option} -- solves \textbf{Non-language-based interaction} \\ Prior model collation \cite{Aghaee2012}

Programming by example modification (PbEM) (also referred to as ``fork and edit'') is where the user acquires composites that have been created already as a base example, and is then able to edit this example to reduce the burden in the composition process. 

\textbf{Language-based interaction} \\ \emph{Decision} -- solves \textbf{Interaction technique} \\ Prior model collation \cite{Minhas2012}

Language-based interactions rely on the user of the tool manipulating the composition by editing some language.

\textbf{Domain-specific Language (DSL)} \\ \emph{Decision} -- solves \textbf{Language-based interaction} \\ Prior model collation 

A DSL is a language that is specialsed so that it can be used more effectively in a particular domain. This decision was first identified in Prior model collation to group together specific instances of domain-specific languages, and was then reinforced in Lit review from the literature \cite{Skrobo2011}.

\textbf{Visual DSL} \\ \emph{Option} -- solves \textbf{DSL} \\ Prior model collation \cite{Grammel2010,Aghaee2012}

A visual DSL is a DSL that doesn't rely solely on text to convey its meaning, it might instead use some form of graphical representation as the syntax to convey the semantics of the language.

\textbf{Textual DSL} \\ \emph{Option} -- solves \textbf{DSL} \\ Prior model collation \cite{Aghaee2012}

A textual DSL is a DSL defined with a textual syntax. This tends to mean that they are expressive, but have similar barriers to learning to full programming languages \cite{Aghaee2012}.

\textbf{Full language} \\ \emph{Option} -- solves \textbf{Language-based interaction} \\ Prior model collation \cite{Fischer2009,Pietschmann2010}

Full programming languages can also be used to perform composition, and are normally text-based. They tend to be more expressive than DSLs because they aren't specialised, but have a much higher learning barrier than DSLs.

\textbf{Blackboard} \\ \emph{Option} -- solves \textbf{Interaction technique} \\ Prior model collation \cite{Aghaee2012}

Blackboard interaction allows multiple users to edit mashups (and hence composition), where all users can see the edits that are being made in real time. This option is orthogonal to the other properties of interaction, it could either language- or non-language-based.

\subsubsection{Representation}

\textbf{Representation} \\ \emph{Decision} \\ Tool review

This decision was created in the tool review to group together the representation of the various different entities in the composition process.

\textbf{Composition representation} \\ \emph{Decision} -- solves \textbf{Representation} \\ Tool review

This decision was also identified in the tool review, to group together the representation of the various aspects of the entities within and process of composition.

\textbf{Component representation} \\ \emph{Decision} -- solves \textbf{Composition representation} \\ Tool review

The representation of components in the composition process can differ from the representation of components outside of this process, but still in the tool. This decision was identified in the tool review to group together the various differnet possible component representations.

\textbf{WYSIWYG component representation} \\ \emph{Option} -- solves \textbf{Component representation} \\ Prior model collation \cite{Aghaee2012}

A WYSIWYG component representation is linked to a WYSIWYG interaction technique, in that in order for the user to interact with the composition process in a WYSIWYG manner, the components must be represented as how they would appear when the composite is executed.

\textbf{Textual component representation} \\ \emph{Option} -- solves \textbf{Component representation} \\ Prior model collation \cite{Aghaee2012,Fischer2009,Pietschmann2010}

A textual component representation means that the component is represented solely with a piece of text. This would normally be the case in a textual DSL/full language reprensentation, although components are often represented with some text to convey, for example, the name of the service could be presented along with some graphical representation of the component. This option was identified in numerous prior DS models \cite{Aghaee2012,Fischer2009,Pietschmann2010}.

\textbf{Iconic component representation} \\ \emph{Option} -- solves \textbf{Component representation} \\ Prior model collation \cite{Aghaee2012}

An iconic representation means that the component is represented using its icon. Note that using an iconic representation does not preclude the inclusion of some other form of representation. This option was identified in the prior DS model created by Aghaee et al. \cite{Aghaee2012}.

\textbf{Form-element representation} \\ \emph{Option} -- solves \textbf{Component representation} \\ Tool review  -- later iteration

Form-element representation is linked to form-based interaction with the composition process, and represent each component as an element within the form. Each component would typically represent its inputs (if any are present) as input elements on the form. For example each input could be representated as a text field. This option was identified in a later iteration of the tool review.

\textbf{Flow diagram component representation} \\ \emph{Option} -- solves \textbf{Component representation} \\ Tool review

Flow diagram components tend to be simple box and line representations, which makes it a very simple representation of a component. This could be viewed as the most simple case of a form-based representation, where the component being represented doesn't have any inputs.

\textbf{``Flow'' representation} \\ \emph{Decision} -- solves \textbf{Composition representation} \\ Prior model collation

Flow representation is the repesentation of what flows between the components, be that data, control, etc. This decision was created to group together the different options for representing what flow is present in the composition process.

\textbf{Control flow representation} \\ \emph{Decision} -- solves \textbf{Flow representation} \\ Prior model collation

Control flow (also known as process flow) is the idea that control moves between the components when the results of the composition is executed. This decision was created to group together the different options for how control flow can be represented in composition. Control flow could also be thought of as the order of the execution, and requires that one component must be executed before another. Control flow allows the representation of branching, looping, etc.

Note that for the composition to make sense, control flow must always be present somewhere in the composition process, although it may not be presented to the user.

\textbf{Explicit control flow representation} \\ \emph{Option} -- solves \textbf{Control flow representation} \\ Prior model collation \cite{Grammel2010,Aghaee2012}

An explicit control flow representation shows the flow of control explicitly to the user.

\textbf{Implicit control flow representation} \\ \emph{Option} -- solves \textbf{Control flow representation} \\ Prior model collation \cite{Aghaee2012}

In the model in which it is suggested, an implicit control flow representation is stated as being the same as a data flow \cite{Aghaee2012}. However, we disagree with this premise since it is entirely possible that a composition representation could present neither control nor data flow.

\textbf{Data flow representation} \\ \emph{Decision} -- solves \textbf{Flow representation} \\ Tool review

Data flow is the passing of data between components, although there is no implication as to the order in which the components are executed \cite{Wajid2010}. Data flow may not always be present, but if it is, there are two options for how it can be represented.

\textbf{Explicit data flow representation} \\ \emph{Option} -- solves \textbf{Data flow representation} \\ Prior model collation \cite{Grammel2010}

An explicit data flow representation explicitly presents the detail of the data that is being passed between the services.

\textbf{Implicit data flow representation} \\ \emph{Option} -- solves \textbf{Data flow representation} \\ Tool review

Implicit data flow representation means that data flow is present in the composition process, but it is not represented explicitly to the user.

\textbf{Component-link representation} \\ \emph{Decision} -- sovles \textbf{Composition representation} \\ Tool review



\textbf{Graphical component links} \\ \emph{Decision} -- solves \textbf{Component-link representation} \\ Tool review

\textbf{Wired component links} \\ \emph{Option} -- solves \textbf{Graphical component links} \\ Prior model collation \cite{Grammel2010,Aghaee2012,Minhas2012,Fischer2009,Pietschmann2010}

%		\item \textit{Wiring} -- Whether the components in the mashup are connected together with line-based connectors (which could represent operations such as filtering, sorting, etc.)
%		\item \textit{Wiring} facilitates the mashup development by supporting connectors between modules, blocks, components (popularly known as widgets, gadgets or badgets)

\textbf{Non-wire graphical component links} \\ \emph{Option} -- solves \textbf{Graphical component links} \\ Tool review

\textbf{Visual DSL links} \\ \emph{Option} -- solves \textbf{Graphical component links} \\ Prior model collation \cite{Grammel2010,Aghaee2012}

\textbf{Textual component links} \\ \emph{Decision} -- solves \textbf{Component-link representation} \\ Tool review

\textbf{Textual DSL component links} \\ \emph{Option} -- solves \textbf{Textual component links} \\ Prior model collation \cite{Aghaee2012}

\textbf{Full language component links} \\ \emph{Option} -- solves \textbf{Textual component links} \\ Prior model collation \cite{Fischer2009}

\textbf{No visual links} \\ \emph{Option} -- solves \textbf{Component-link representation} \\ Tool review
		
\textbf{Composition process representation} \\ \emph{Decision} -- solves \textbf{Composition representation} \\ Prior model collation

This decision was created to group together the various options for representing the composition process as a whole.

\textbf{Flow diagram representation} \\ \emph{Option} -- solves \textbf{Composition process representation} \\ Tool review

\textbf{Spreadsheet representation} \\ \emph{Option} -- solves \textbf{Composition process representation} \\ Prior model collation \cite{Aghaee2012,Minhas2012,Fischer2009}

%		\item \textit{Spreadsheet} enables the data to be loaded into tables and then process into the desired format. It helps the end-users by eliminating much of the programming effort and allows users to see the initial and desired state of the data.

\textbf{Form representation} \\ \emph{Option} -- solves \textbf{Composition process representation} \\ Prior model collation \cite{Aghaee2012}

\textbf{Page flow representation} \\ \emph{Option} -- solves \textbf{Composition process representation} \\ Prior model collation \cite{Pietschmann2010}

\textbf{Plain list representation} \\ \emph{Option} -- solves \textbf{Composition process representation} \\ Tool review  -- later iteration

\textbf{Text pane/editor representation} \\ \emph{Option} -- solves \textbf{Composition process representation} \\ Tool review

\textbf{Abstraction} \\ \emph{Decision} -- solves \textbf{Representation} \\ Tool review  -- later iteration

\textbf{Abstraction level} \\ \emph{Decision} -- solves \textbf{Abstraction} \\ Prior model collation \cite{Grammel2010}

%\item \textbf{High} -- no knowledge is required, but flexibility is restricted to reusing and confusing mashups that are developed by others. Working with a high level of abstraction is supported by mashup development environments in several ways:
%	\begin{itemize*}
%		\item Reuse of complete mashups created by others
%		\item High-level mashup and widget parameterisation
%		\item Automatic reuse of data extractors for websites
%		\item Programming by Example
%	\end{itemize*}
%	\item \textbf{Intermediate} -- knowledge about concepts such as data flow, data types or UI widgets is required, but the technological details of these concepts are hidden. Intermediate abstraction is normally supported via visual Domain Specific Languages (DSLs):
%	\begin{itemize*}
%		\item Visual dataflow languages
%		\item Visual workflow/process orchestration languages
%		\item Dialog-based wiring of widgets
%	\end{itemize*}
%	\item \textbf{Low} -- programming knowledge required, but maximum flexibility
%	\begin{itemize*}
%		\item Textual DSL editors
%		\item Textual DSLs in dialog fields
%		\item Providing extension APIs
%	\end{itemize*} 

\textbf{High abstraction level} \\ \emph{Option} -- solves \textbf{Abstraction level} \\ Prior model collation \cite{Grammel2010}

\textbf{Average abstraction level} \\ \emph{Option} -- solves \textbf{Abstraction level} \\ Prior model collation \cite{Grammel2010}

\textbf{Low abstraction level} \\ \emph{Option} -- solves \textbf{Abstraction level} \\ Prior model collation \cite{Grammel2010}

\textbf{No abstraction} \\ \emph{Option} -- solves \textbf{Abstraction level} \\ Lit review \cite{Nestler2010}

\textbf{Metaphor} \\ \emph{Option} -- solves \textbf{Abstraction} \\ Requirements gathering study
 
\textbf{Aesthetics} \\ \emph{Decision} -- solves \textbf{Representation} \\ Requirements gathering study

\textbf{Colour} \\ \emph{Decision} -- solves \textbf{Aesthetics} \\ Requirements gathering study

This is a decision because colour was mentioned in our requirements gathering study, but none of the participants suggested any good ways of sorting it out

\textbf{Visual} \\ \emph{Decision} -- solves \textbf{Aesthetics} \\ Requirements gathering study

PArticpants in our study indicated that composition should be a very visual process, but we don't have any options to suggest how this should proceeed.

\subsubsection{Collaboration}

\textbf{Collaboration} \\ \emph{Decision} \\ Prior model collation \cite{Aghaee2012}

\textbf{Community} \\ \emph{Decision} -- solves \textbf{Collaboration} \\ Prior model collation

This decision was created to group together various aspects of the community provided by and associated with the tool.

\textbf{Community support} \\ \emph{Decision} -- solves \textbf{Community} \\ Prior model collation \cite{Aghaee2012}

\textbf{Discussion forum} \\ \emph{Option} -- solves \textbf{Community support} \\ Prior model collation \cite{Grammel2010}

\textbf{Wiki} \\ \emph{Option} -- solves \textbf{Community support} \\ Prior model collation \cite{Aghaee2012}

%		\item \textit{Wiki} -- This utilises the Wiki method to create mashups, where users are able to work on a mashup simultaneously, with associated tools to keep track of edits and versioning history.

\textbf{Social Networking Service (SNS) Support} \\ \emph{Option} -- solves \textbf{Community support} \\ Prior model collation \cite{Grammel2010}

\textbf{Blog} \\ \emph{Option} -- solves \textbf{Community support} \\ Tool review

\textbf{Online community access} \\ \emph{Decision} -- solves \textbf{Community} \\ Stage \cite{Aghaee2012}

%	\item \textbf{Online Community.} These would typically be used by end users in order to get assistance while they are creating mashups, and are classified by the level of security that they provide.

\textbf{Public acccess} \\ \emph{Option} -- solves \textbf{Online community access} \\ Prior model collation \cite{Aghaee2012}

%		\item \textit{Public} -- These can be accessed by any user who wishes to join (note that registration may still be required)

\textbf{Private access} \\ \emph{Option} -- solves \textbf{Online community access} \\ Prior model collation \cite{Aghaee2012}

%		\item \textit{Private} -- These would require some barrier for entry to the community, e.g. if it were invite only, or solely for enterprise users.

\subsubsection{Target User}

%	\item \textbf{Target end-user.} This is the range of users' technical skills. It ranges from non-programmer to programmer, with a space in the middle for technical-non-programmers, which have been defined as ``local users''.
%	\begin{itemize*}
%		\item \textit{Non-programmers} -- Tools are needed which require minimal technical skills. They need tools which limit what they can perform, and ones which provide assistance to them. {\color{red} this is a bad name for a user group}
%		\item \textit{Local Developers}  -- Non-programmers who have advanced knowledge of technical tools, they are willing to explore and harvest all of the functionality of a mashup tool tailored for their abilities. Technical complexity may still need to be hidden from them.
%		\item \textit{Programmers} -- can program. Tools targeting programmers can provide very high levels of customisation and freedom.
%	\end{itemize*}

\textbf{Target user} \\ \emph{Decision} \\ Prior model collation

Group

\textbf{User expertise} \\ \emph{Decision} -- solves \textbf{Target user} \\ Prior model collation

Group

\textbf{Technical expertise} \\ \emph{Decision} -- solves \textbf{User expertise} \\ Lit review \cite{Silva2008}

\textbf{Technical expert} \\ \emph{Option} -- solves \textbf{Technical expertise} \\ Lit review \cite{Silva2008,Pietschmann2010}

\textbf{Non-technical user} \\ \emph{Option} -- solves \textbf{Technical expertise} \\ Lit review \cite{Silva2008,Pietschmann2010}

\textbf{Programming expertise} \\ \emph{Decision} -- solves \textbf{Technical expertise} \\ Prior model collation \cite{Aghaee2012,Patel2010}

\textbf{Non-programmer} \\ \emph{Option} -- solves \textbf{Programming expertise} \\ Prior model collation \cite{Aghaee2012,Patel2010,Pietschmann2010}

\textbf{Programmer} \\ \emph{Option} -- solves \textbf{Programming expertise} \\ Prior model collation \cite{Aghaee2012,Patel2010,Pietschmann2010}

\textbf{Domain expertise} \\ \emph{Decision} -- solves \textbf{User expertise} \\ Lit review \cite{Silva2008}

\textbf{Domain newbie} \\ \emph{Option} -- solves \textbf{Domain expertise} \\ Lit review \cite{Silva2008}

\textbf{Domain expert} \\ \emph{Option} -- solves \textbf{Domain expertise} \\ Lit review \cite{Silva2008}

\textbf{User (business) context} \\ \emph{Decision} -- solves \textbf{Target user} \\ Prior model collation \cite{Minhas2012}

\textbf{Consumer/individual} \\ \emph{Option} -- solves \textbf{User context} \\ Prior model collation \cite{Minhas2012,Patel2010}

\textbf{Enterprise/organisation} \\ \emph{Option} -- solves \textbf{User context} \\ Prior model collation \cite{Minhas2012,Patel2010}

\textbf{Learning curve} \\ \emph{Decision} -- solves \textbf{Target user} \\ Prior model collation \cite{Minhas2012}

%	\item \textit{Learning curve} Also known as learnability, this feature indicates how easy it is to learn a particular tool. The evaluation for this feature is done on a 3-point scale: high, medium, low. The easier it is to learn a particular tool, the higher the learnability and vice versa.

\textbf{Low learning curve} \\ \emph{Option} -- sovles \textbf{Learning curve} \\ Prior model collation \cite{Minhas2012}

\textbf{Medium learning curve} \\ \emph{Option} -- sovles \textbf{Learning curve} \\ Prior model collation \cite{Minhas2012}

\textbf{High learning curve} \\ \emph{Option} -- sovles \textbf{Learning curve} \\ Prior model collation \cite{Minhas2012}

\subsubsection{Target Domain}

\textbf{Domain/context specificity} \\ \emph{Decision} \\ Prior model collation \cite{Aghaee2012}

%\textbf{Specificity.} Some design tools work across multiple domains, others are constrained to specific domains

\textbf{Generic} \\ \emph{Option} -- solves \textbf{Domain/context specificity} \\ Prior model collation \cite{Aghaee2012}

%\item \textit{Generic} -- They do not target any specific group of users or any particular domain, but instead focus on the day-to-day needs of end-users. These tend to reach a wide range of users, although it is not trivial to cater for all needs and abilities. 

\textbf{Context-specific} \\ \emph{Option} -- solves \textbf{Domain/context specificity} \\ Tool review

\textbf{Domain-specific} \\ \emph{Option} -- solves \textbf{Domain/context specificity} \\ Tool review

\textbf{Specialised} \\ \emph{Option} -- solves \textbf{Domain/context specificity} \\ Prior model collation \cite{Aghaee2012}

%		\item \textit{Specialised} -- Mashups target specific domains, e.g. e-learning or telecommunications. Users of such tools tend to have similar needs and technical abilities, meaning tools can be better tailored to their needs. However, the mashup tool designer also needs in-depth knowledge of the area.

\subsubsection{Execution Context}

\textbf{Execution context} \\ \emph{Decision} \\ Tool review

\textbf{Mobile context} \\ \emph{Option} -- solves \textbf{Execution context} \\ Tool review

\textbf{Desktop context} \\ \emph{Option} -- solves \textbf{Execution context} \\ Tool review

\textbf{Web context} \\ \emph{Option} -- solves \textbf{Execution context} \\ Tool review

\subsubsection{Usability}

\textbf{Usability} \\ \emph{Decision} \\ Prior model collation \cite{Aghaee2012}

\textbf{Prompts/suggesetions} \\ \emph{Decision} -- solves \textbf{Usability} \\ Prior model collation \cite{Minhas2012}

\textbf{Incompatibility promtpts} \\ \emph{Option} -- solves \textbf{Prompts/suggestions} \\ Prior model collation \cite{Minhas2012}

%	\item \textit{Signal/prompt incompatible situations/provide argumentation} A component that monitors the user's work and offers suggestions for changes.

\textbf{Suggestions} \\ \emph{Option} -- solves \textbf{Prompts/suggestions} \\ Prior model collation \cite{Minhas2012}

\textbf{Internal help systems} \\ \emph{Decision} -- solves \textbf{Usability} \\ Prior model collation

Group

\textbf{Help} \\ \emph{Option} -- solves \textbf{Internal help systems} \\ Prior model collation \cite{Grammel2010}

\textbf{User goal/requirement mapping} \\ \emph{Option} -- solves \textbf{Internal help systems} \\ Prior model collation \cite{Minhas2012}

%		\item \textit{User Goals} The feature evaluates the explicit or even implicit support of user goals or tasks by the system/tool under review where the goal defines any objective in user terms such as arranging a hang out with friends that might consist of several low-level requirements.
%		\item \textit{User Requirements} Technical jargon intimidates users from developing mashups or composing services through tools this criterion will determine whether tools specify the user requirements beyond listing the APIs operations and other technical descriptions.

\textbf{Version control} \\ \emph{Option} -- solves \textbf{Internal help systems} \\ Prior model collation \cite{Grammel2010}

\textbf{Scenarios} \\ \emph{Option} -- solves \textbf{Internal help systems} \\ Requirements gathering study

\textbf{External help systems} \\ \emph{Decision} -- solves \textbf{Usability} \\ Tool review  -- later iteration

\textbf{Documentation} \\ \emph{Decision} -- solves \textbf{External help systems} \\ Prior model collation \cite{Grammel2010}

\textbf{Tool documentation} \\ \emph{Option} -- solves \textbf{Documentation} \\ Prior model collation \cite{Grammel2010}

\textbf{API/component documentation} \\ \emph{Option} -- solves \textbf{Documentation} \\ Prior model collation \cite{Grammel2010}

\textbf{Screencasts} \\ \emph{Option} -- solves \textbf{Documentation} \\ Tool review

\textbf{Discussion forums} \\ \emph{Option} -- solves \textbf{External help systems} \\ Prior model collation \cite{Grammel2010}

\textbf{External help/FAQ} \\ \emph{Option} -- solves \textbf{External help systems} \\ Tool review

\textbf{Tutorials} \\ \emph{Option} -- solves \textbf{Usability} \\ Prior model collation \cite{Grammel2010,Minhas2012}

%	\item \textit{Tutorial element} An embedded tutorial element that assists the user in learning about the application and the domain

\textbf{Re-use} \\ \emph{Decision} -- solves \textbf{Usability} \\ Prior model collation \cite{Grammel2010}

\textbf{Examples} \\ \emph{Option} -- solves \textbf{Re-use} \\ Prior model collation \cite{Grammel2010}

\subsection{Structural Sub-model}

\begin{figure}[htb]
	\centering
	\includegraphics[width=\textwidth,height=\textheight,keepaspectratio]{images/ds/structural_submodel_1}
	\caption{The first part of the structural DS sub-model}
	\label{ds:fig:structural1}
\end{figure}

\begin{figure}[htb]
	\centering
	\includegraphics[width=\textwidth,height=\textheight,keepaspectratio]{images/ds/structural_submodel_2}
	\caption{The second part of the structural DS sub-model}
	\label{ds:fig:structural2}
\end{figure}

\clearpage

\subsubsection{Composition Structure}

\textbf{Composition structure} \\ \emph{Decision} \\ Prior model collation

Group

\textbf{Liveness} \\ \emph{Decision} -- solves \textbf{Composition structure} \\ Prior model collation \cite{Aghaee2012}

%	\item \textbf{Liveness.} Liveness is a concept from visual language which provides four levels which link the visual language to the output of the creation process. Aghaee et al. suggest it is applicable to mashups too \cite{Aghaee2012}

\textbf{Liveness level 1} \\ \emph{Option} -- solves \textbf{Liveness} \\ Prior model collation \cite{Aghaee2012}

%\textit{Level 1: Flowchart as ancillary description} -- Tools are just used to create a prototypical description, which is not directly connected to any sort of execution engine. These tend to be simpler, but limit the mechanisms by which the user can validate the mashup (i.e. no testing).

\textbf{Liveness level 2} \\ \emph{Option} -- solves \textbf{Liveness} \\ Prior model collation \cite{Aghaee2012}

%\textit{Level 2: Executable flowchart} -- This level of liveness indicates that the flowchart description contains executable semantics. These can be converted into a form which could be executed, but require additional technical details.

\textbf{Liveness level 3} \\ \emph{Option} -- solves \textbf{Liveness} \\ Prior model collation \cite{Aghaee2012}

%\textit{Level 3: Edit triggered updates} -- This is the level where design and run-time are both available to the user of the mashup tool. The problem with this view is that it can be difficult for users to be able to distinguish between design-time and runtime

\textbf{Liveness level 4} \\ \emph{Option} -- solves \textbf{Liveness} \\ Prior model collation \cite{Aghaee2012}

%\textit{Level 4: Stream-driven updates} -- This level allows for the live editing of mashup code whilst it is being executed. Changes are instantly observable, but executing changes in this way increases the chances of failure.

\textbf{Programming structures} \\ \emph{Decision} -- solves \textbf{Composition structure} \\ Tool review

\textbf{Branches} \\ \emph{Option} -- solves \textbf{Programming structures} \\ Tool review

\textbf{Loops} \\ \emph{Option} -- solves \textbf{Programming structures} \\ Tool review

\textbf{Component types} \\ \emph{Decision} -- solves \textbf{Composition structure} \\ Tool review

\textbf{Triggers} \\ \emph{Option} -- solves \textbf{Component types} \\ Tool review

\textbf{Actions} \\ \emph{Option} -- solves \textbf{Component types} \\ Tool review

\textbf{Filters} \\ \emph{Option} -- solves \textbf{Component types} \\ Tool review

\textbf{Pervasive services} \\ \emph{Option} -- solves \textbf{Component types} \\ Requirements gathering study

\textbf{Technologies} \\ \emph{Decision} -- solves \textbf{Composition structure} \\ Prior model collation

Group

\textbf{Communication protocol} \\ \emph{Decision} -- solves \textbf{Technologies} \\ Prior model collation \cite{Minhas2012}

%	\item \textit{Protocols supported for communication with Web Services} For communication with web services, different protocols and standards have been defined that deal with the message passing.

\textbf{SOAP} \\ \emph{Option} -- solves \textbf{Communication protocol} \\ Prior model collation \cite{Minhas2012}

\textbf{REST} \\ \emph{Option} -- solves \textbf{Communication protocol} \\ Prior model collation \cite{Minhas2012}

\textbf{Data retrieval} \\ \emph{Decision} -- solves \textbf{Technologies} \\ Prior model collation \cite{Minhas2012}

%	\item \textit{Data retrieval strategy} One important step in mashup development is the access to the data that is intended to be mashed up. While the composition tools normally rely on widgets for this purpose, there are other scraping tools that directly access the web pages through their DOM.

\textbf{Syndication format} \\ \emph{Decision} -- solves \textbf{Data retrieval} \\ Prior model collation \cite{Minhas2012}

%	\item \textit{Syndication formats supported} Different websites publish their regular content using different syndication formats

\textbf{RSS} \\ \emph{Option} -- solves \textbf{Syndication format} \\ Prior model collation \cite{Minhas2012}

\textbf{ATOM} \\ \emph{Option} -- solves \textbf{Syndication format} \\ Prior model collation \cite{Minhas2012}

\textbf{Widget-based data retrieval} \\ \emph{Option} -- solves \textbf{Data retrieval} \\ Prior model collation \cite{Minhas2012}

\textbf{Web scraping} \\ \emph{Option} -- solves \textbf{Data retrieval} \\ Prior model collation \cite{Minhas2012}

\textbf{Message data format} \\ \emph{Decision} -- solves \textbf{Technologies} \\ Prior model collation \cite{Pietschmann2010}

\textbf{JSON messages} \\ \emph{Option} -- solves \textbf{Message data format} \\ Prior model collation \cite{Pietschmann2010}

\textbf{XML messages} \\ \emph{Option} -- solves \textbf{Message data format} \\ Prior model collation \cite{Pietschmann2010}

\textbf{Service description format} \\ \emph{Decision} -- solves \textbf{Technologies} \\ Prior model collation \cite{Pietschmann2010}

\textbf{JSON description} \\ \emph{Option} -- solves \textbf{Service description format} \\ Prior model collation \cite{Pietschmann2010}

\textbf{XML description} \\ \emph{Option} -- solves \textbf{Service description format} \\ Prior model collation \cite{Pietschmann2010}

\textbf{WSDL description} \\ \emph{Option} -- solves \textbf{Service description format} \\ Prior model collation \cite{Pietschmann2010}

\textbf{WADL description} \\ \emph{Option} -- solves \textbf{Service description format} \\ Prior model collation \cite{Pietschmann2010}

\textbf{Templates} \\ \emph{Decision} -- solves \textbf{Composition structure} \\ Prior model collation \cite{Pietschmann2010}

\textbf{Template ``level''} \\ \emph{Decision} -- solves \textbf{Templates} \\ Requirements gathering study

\textbf{Composition-level template} \\ \emph{Option} -- solves \textbf{Template level} \\ Prior model collation \cite{Pietschmann2010}

\textbf{Tool-level template} \\ \emph{Option} -- solves \textbf{Template level} \\ Prior model collation \cite{Pietschmann2010}

\textbf{Linear template} \\ \emph{Option} -- solves \textbf{Templates} \\ Requirements gathering study

\textbf{No template} \\ \emph{Option} -- solves \textbf{Templates} \\ Requirements gathering study

\textbf{Automation degree} \\ \emph{Decision} -- solves \textbf{Composition structure} \\ Prior model collation \cite{Aghaee2012}

%	\item \textbf{Automation Degree.} This refers to how much of the development process can be be undertaken by the tool on behalf of its users.
%		\item \textit{Full automation} -- The user is not directly involved in the development process, they instead provide the input to the process, supervise, and validate the output.
%		\item \textit{Semi-automation} -- These tools partially automate the mashup process by providing assistance and guidance.

\textbf{Full automation} \\ \emph{Option} -- solves \textbf{Automation degree} \\ Prior model collation \cite{Aghaee2012,Fischer2009}

\textbf{Semi-automation} \\ \emph{Option} -- solves \textbf{Automation degree} \\ Prior model collation \cite{Aghaee2012,Fischer2009}

\textbf{No automation} \\ \emph{Option} -- solves \textbf{Automation degree} \\ Prior model collation \cite{Fischer2009}

\textbf{Composition layer} \\ \emph{Decision} -- solves \textbf{Composition structure} \\ Lit review \cite{Paterno2011}

\textbf{Service layer} \\ \emph{Option} -- solves \textbf{Composition layer} \\ Lit review \cite{Paterno2011}

\textbf{Application layer} \\ \emph{Option} -- solves \textbf{Composition layer} \\ Lit review \cite{Paterno2011}

\textbf{Presentation layer} \\ \emph{Option} -- solves \textbf{Composition layer} \\ Lit review \cite{Paterno2011}

\textbf{``Flow'' type supported} \\ \emph{Decision} -- solves \textbf{Composition structure} \\ Prior model collation

Group

\textbf{Data flow supported} \\ \emph{Option} -- solves \textbf{Flow type supported} \\ Prior model collation \cite{Grammel2010,Pietschmann2010}

\textbf{Control flow supported} \\ \emph{Option} -- solves \textbf{Flow type supported} \\ Prior model collation \cite{Grammel2010,Pietschmann2010}

\textbf{Inputs \& outputs} \\ \emph{Decision} -- solves \textbf{Composition structure} \\ Tool review

\textbf{Data types} \\ \emph{Option} -- solves \emph{Inputs \& outputs} \\ Tool review  -- later iteration

\textbf{Multiple atomic inputs, outputs} \\ \emph{Option} -- solves \emph{Inputs \& outputs} \\ Tool review

\textbf{Single composite inputs, outputs} \\ \emph{Option} -- solves \emph{Inputs \& outputs} \\ Tool review

\textbf{Infinite composition} \\ \emph{Option} -- solves \textbf{Composition structure} \\ Requirements gathering study

\subsubsection{Tool Structure}

\textbf{Tool structure} \\ \emph{Decision} \\ Prior model collation

Group

\textbf{System integration} \\ \emph{Option} -- solves \textbf{Tool structure} \\ Requirements gathering study

\textbf{Extension APIs} \\ \emph{Option} -- solves \textbf{Tool structure} \\ Prior model collation \cite{Grammel2010}

\textbf{Application type} \\ \emph{Decision} -- solves \textbf{Tool structure} \\ Prior model collation \cite{Minhas2012}

\textbf{Application} \\ \emph{Option} -- solves \textbf{Application type} \\ Tool review

\textbf{Plug-in} \\ \emph{Option} -- solves \textbf{Application type} \\ Tool review

\textbf{Platform} \\ \emph{Option} -- solves \textbf{Application type} \\ Tool review

\textbf{Web page/webapp} \\ \emph{Option} -- solves \textbf{Application type} \\ Tool review

\textbf{Framework} \\ \emph{Option} -- solves \textbf{Application type} \\ Prior model collation \cite{Minhas2012}

\textbf{Application location} \\ \emph{Decision} -- solves \textbf{Tool structure} \\ Prior model collation \cite{Bronsted2010a}

\textbf{Mobile} \\ \emph{Option} -- solves \textbf{Application location} \\ Tool review

\textbf{Desktop} \\ \emph{Option} -- solves \textbf{Application location} \\ Tool review

\textbf{Web} \\ \emph{Option} -- solves \textbf{Application location} \\ Tool review

\textbf{Repositories} \\ \emph{Decision} -- solves \textbf{Tool structure} \\ Prior model collation \cite{Pietschmann2010}

\textbf{Composite repository} \\ \emph{Option} -- solves \textbf{Repositories} \\ Prior model collation \cite{Pietschmann2010}

\textbf{Component repository} \\ \emph{Option} -- solves \textbf{Repositories} \\ Prior model collation \cite{Pietschmann2010}

\textbf{Infrastructure} \\ \emph{Decision} -- solves \textbf{Tool structure} \\ Prior model collation \cite{Bronsted2010a}

\textbf{Ad hoc infrastructure} \\ \emph{Option} -- solves \textbf{Infrastructure} \\ Prior model collation \cite{Bronsted2010a}

\textbf{Fixed infrastructure} \\ \emph{Option} -- solves \textbf{Infrastructure} \\ Prior model collation \cite{Bronsted2010a}

\textbf{Topology} \\ \emph{Decision} -- solves \textbf{Tool structure} \\ Prior model collation \cite{Bronsted2010a}

\textbf{Centralised topology} \\ \emph{Option} -- solves \textbf{Topology} \\ Prior model collation \cite{Bronsted2010a}

\textbf{Decentralised topology} \\ \emph{Option} -- solves \textbf{Topology} \\ Prior model collation \cite{Bronsted2010a}

\textbf{Sharable entities} \\ \emph{Decision} -- solves \textbf{Tool structure} \\ Prior model collation \cite{Grammel2010}

\textbf{Sharable components} \\ \emph{Option} -- solves \textbf{Sharable entities} \\ Prior model collation \cite{Grammel2010}

\textbf{Sharable composites} \\ \emph{Option} -- solves \textbf{Sharable entities} \\ Prior model collation \cite{Grammel2010}

\textbf{Customisation} \\ \emph{Decision} -- solves \textbf{Tool structure} \\ Requirements gathering study

\subsection{Service Sub-model}

\begin{figure}[htb]
	\centering
	\includegraphics[width=\textwidth,height=\textheight,keepaspectratio]{images/ds/service_submodel}
	\caption{The service DS sub-model}
	\label{ds:fig:service}
\end{figure}

\clearpage

\subsubsection{Service Interactions}

% Put the both attributes as a separate thing to make explaining them easier

\textbf{Service interactions} \\ \emph{Decision} \\ Tool review

\textbf{Rate service} \\ \emph{Option} -- solves \textbf{Service interactions} \\ Prior model collation \cite{Grammel2010}

\textbf{Review service} \\ \emph{Option} -- solves \textbf{Service interactions} \\ Tool review  -- later iteration

\textbf{Tag service} \\ \emph{Option} -- solves \textbf{Service interactions} \\ Prior model collation \cite{Grammel2010}

\textbf{Composite interactions} \\ \emph{Decision} -- solves {Service interactions} \\ Tool review

\textbf{Paramters} \\ \emph{Decision} -- solves {Composite interactions} \\ Requirements gathering study

\textbf{Set paramters at composition time} \\ \emph{Decision} -- solves {Parameters} \\ Requirements gathering study

\textbf{Set paramters at runtime} \\ \emph{Decision} -- solves {Parameters} \\ Requirements gathering study

\textbf{Assign tags to composite} \\ \emph{Option} -- solves \textbf{Composite interactions} \\ Tool review  -- later iteration

\textbf{Upload/publish composite} \\ \emph{Option} -- solves \textbf{Composite interactions} \\ Tool review

\textbf{Copy composite} \\ \emph{Option} -- solves \textbf{Composite interactions} \\ Tool review

\textbf{Export composite} \\ \emph{Option} -- solves \textbf{Composite interactions} \\ Tool review

\textbf{Rename composite} \\ \emph{Option} -- solves \textbf{Composite interactions} \\ Tool review

\textbf{Save composite} \\ \emph{Option} -- solves \textbf{Composite interactions} \\ Tool review

\textbf{Delete composite} \\ \emph{Option} -- solves \textbf{Composite interactions} \\ Tool review

\textbf{Turn on/off composite} \\ \emph{Option} -- solves \textbf{Composite interactions} \\ Tool review

\textbf{Acquire composite} \\ \emph{Option} -- solves \textbf{Composite interactions} \\ Tool review

\textbf{Sharre composite on SNS} \\ \emph{Option} -- solves \textbf{Composite interactions} \\ Tool review

\textbf{Edit composite} \\ \emph{Option} -- solves \textbf{Composite interactions} \\ Tool review

\textbf{Execute composite} \\ \emph{Option} -- solves \textbf{Composite interactions} \\ Tool review

\textbf{Choose composite icon} \\ \emph{Option} -- solves \textbf{Composite interactions} \\ Tool review

\textbf{Component interactions} \\ \emph{Decision} -- solves {Service interactions} \\ Tool review

\textbf{Activate component} \\ \emph{Option} -- solves \textbf{Component interactions} \\ Tool review

\textbf{Set/edit component parameters} \\ \emph{Option} -- solves \textbf{Component interactions} \\ Tool review

\textbf{Use component in composition} \\ \emph{Option} -- solves \textbf{Component interactions} \\ Tool review

\textbf{View example usage for component} \\ \emph{Option} -- solves \textbf{Component interactions} \\ Tool review  -- later iteration

\textbf{Import component} \\ \emph{Option} -- solves \textbf{Component interactions} \\ Tool review  -- later iteration

\subsubsection{Service Attributes}

\textbf{Service attributes} \\ \emph{Decision} \\ Tool review

\textbf{Service functional attributes} \\ \emph{Decision} -- solves \textbf{Service attributes} \\ Tool review

\textbf{Service name} \\ \emph{Option} -- solves \textbf{Service functional attributes} \\ Tool review

\textbf{Service tags} \\ \emph{Option} -- solves \textbf{Service functional attributes} \\ Tool review

\textbf{Service description} \\ \emph{Option} -- solves \textbf{Service functional attributes} \\ Tool review

\textbf{Service category} \\ \emph{Option} -- solves \textbf{Service functional attributes} \\ Tool review

\textbf{Service popularity attributes} \\ \emph{Decision} -- solves \textbf{Service attributes} \\ Tool review

\textbf{Service number of ratings} \\ \emph{Option} -- solves \textbf{Service popularity attributes} \\ Tool review  -- later iteration

\textbf{Service number of uses} \\ \emph{Option} -- solves \textbf{Service popularity attributes} \\ Tool review

\textbf{Service age} \\ \emph{Option} -- solves \textbf{Service popularity attributes} \\ Tool review

\textbf{Service number of shares on SNS} \\ \emph{Option} -- solves \textbf{Service popularity attributes} \\ Tool review

\textbf{Service number of downloads} \\ \emph{Option} -- solves \textbf{Service popularity attributes} \\ Tool review

\textbf{Service reviews} \\ \emph{Option} -- solves \textbf{Service popularity attributes} \\ Tool review

\textbf{Service ratings} \\ \emph{Option} -- solves \textbf{Service popularity attributes} \\ Tool review

\textbf{Service number of users} \\ \emph{Option} -- solves \textbf{Service popularity attributes} \\ Tool review  -- later iteration

\textbf{Service other attributes} \\ \emph{Decision} -- solves \textbf{Service attributes} \\ Tool review

\textbf{Service creator} \\ \emph{Option} -- solves \textbf{Service other attributes} \\ Tool review

\textbf{Service notes} \\ \emph{Option} -- solves \textbf{Service other attributes} \\ Tool review

\textbf{Service location} \\ \emph{Option} -- solves \textbf{Service other attributes} \\ Tool review

\textbf{Service version} \\ \emph{Option} -- solves \textbf{Service other attributes} \\ Tool review

\textbf{Service copyright} \\ \emph{Option} -- solves \textbf{Service other attributes} \\ Tool review

\textbf{Component attributes} \\ \emph{Decision} -- solves \textbf{Service attributes} \\ Tool review

\textbf{Component functional attributes} \\ \emph{Decision} -- solves \textbf{Component attributes} \\ Tool review

\textbf{Component type} \\ \emph{Option} -- solves \textbf{Component functional attributes} \\ Tool review  -- later iteration

\textbf{Component example usage} \\ \emph{Option} -- solves \textbf{Component functional attributes} \\ Tool review

\textbf{Component options} \\ \emph{Option} -- solves \textbf{Component functional attributes} \\ Tool review

\textbf{Component outputs/results} \\ \emph{Option} -- solves \textbf{Component functional attributes} \\ Tool review

\textbf{Component inputs} \\ \emph{Option} -- solves \textbf{Component functional attributes} \\ Tool review

\textbf{Component requirements/preconditions} \\ \emph{Option} -- solves \textbf{Component functional attributes} \\ Tool review

\textbf{Component icon} \\ \emph{Option} -- solves \textbf{Component functional attributes} \\ Tool review

\textbf{Component other attributes} \\ \emph{Decision} -- solves \textbf{Component attributes} \\ Tool review

\textbf{Component related actions} \\ \emph{Option} -- solves \textbf{Component other attributes} \\ Tool review

\textbf{Composite attributes} \\ \emph{Decision} -- solves \textbf{Service attributes} \\ Tool review

\textbf{Composite functional attributes} \\ \emph{Decision} -- solves \textbf{Composite attributes} \\ Tool review

\textbf{Component names} \\ \emph{Option} -- solves \textbf{Composite functional attributes} \\ Tool review

\textbf{Component icons} \\ \emph{Option} -- solves \textbf{Composite functional attributes} \\ Tool review

\textbf{Component descriptions} \\ \emph{Option} -- solves \textbf{Composite functional attributes} \\ Tool review

\textbf{Composite other attributes} \\ \emph{Decision} -- solves \textbf{Composite attributes} \\ Tool review

\textbf{Composite properties} \\ \emph{Option} -- solves \textbf{Composite other attributes} \\ Tool review

\textbf{Composite icon} \\ \emph{Option} -- solves \textbf{Composite other attributes} \\ Tool review

\textbf{Composite execution status} \\ \emph{Option} -- solves \textbf{Composite other attributes} \\ Requirements gathering study